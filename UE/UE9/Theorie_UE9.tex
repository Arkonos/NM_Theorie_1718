% !TeX spellcheck = de_AT_frami
\documentclass[10pt,a4paper]{article}
\usepackage[utf8x]{inputenc}
\usepackage{ucs}
\usepackage{amsmath}
\usepackage{setspace}
\usepackage{amsfonts}
\usepackage{amssymb}
\usepackage{graphicx}
\usepackage{txfonts}
\usepackage[dvipsnames]{xcolor}
\usepackage{geometry}
\usepackage{graphicx}
\usepackage{epstopdf}
\epstopdfsetup{update}
\geometry{margin= 2cm}
\usepackage[makeroom]{cancel}
\usepackage{multirow}
\setlength\parindent{0pt}

\usepackage{array}
\usepackage{mathtools} %for minus signes in matrices

\usepackage{amsmath,amsfonts,amssymb} % For math equations, theorems, sy
\makeatletter
\def\zz\ignorespaces{\@ifnextchar-{}{\phantom{-}}}
\newcolumntype{R}{>{\zz}{r}}
\makeatother


\begin{document}
	\pagenumbering{gobble}
	\section*{9. Numerik Übungen 2017/18}
	\paragraph{T15}\mbox{}\\
	\textbf{%
		Es sei $\mathbf{R = [2,6]\times[1,3]}$ ein Rechteck. Berechnen Sie das Integral $\mathbf{\iint\limits_R\frac{x}{y}dF}$, indem Sie in x- bzw. y- Richtung die Simpsonregel verwenden. Machen Sie auch eine Skizze (Rechteck, Knoten, Gewichte).
	}\\


	\paragraph{T16}\mbox{}\\
	\textbf{%
	Die Knoten $\mathbf{(0,0), (1,0), (0,1)}$ und die Gewichte $\mathbf{\frac{1}{6},\frac{1}{6},\frac{1}{6}}$ definieren eine Quadraturformel auf dem Einheitsdreieck $\mathbf{D=\left\lbrace (\xi,\eta) \in \mathbb{R}^2 \,;\, 0\leq\xi\leq 1,\, 0\leq\eta \leq1-\xi \right\rbrace}$.
	}

	\paragraph{T17}\mbox{}\\
	\textbf{%
		a) Berechnen Sie die LU-Zerlegung der Matrix A mit Spaltenpivotsuche für
		\begin{align*}\mathbf{
			A= \begin{bmatrix*}[R]
				0   & 2   & 6  & 10 \\
				3   & 6   & 9  & -3 \\
				15  & 0   & 0  & 1  \\
				-15 & -10 & -3 & 1
			\end{bmatrix*}.}
		\end{align*}
	}\\
	\textbf{%
		b) Geben Sie auch die Permutationsmatrix P bzw. den Vektor \emph{IP}. \emph{IP(4)} soll das Vorzeichen für die Berechnung der Determinante erhalten.
	}\\
	\textbf{%
		c) Berechnen Sie mit Hilfe der LU-Zerlegung die Determinante von A.
	}\\
	\textbf{%
		d) Lösen Sie $\mathbf{Ax=b}$ für $\mathbf{b = \begin{bmatrix} 4 \\ -6 \\ -14  \\ -8 \end{bmatrix}}\,$ mit Hilfe der LU-Zerlegung.
	}
\end{document}
