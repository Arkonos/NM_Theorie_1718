% !TeX spellcheck = de_AT_frami
\documentclass[10pt,a4paper]{article}
\usepackage[utf8x]{inputenc}
\usepackage{ucs}
\usepackage{amsmath}
\usepackage{setspace}
\usepackage{amsfonts}
\usepackage{amssymb}
\usepackage{graphicx}
\usepackage{txfonts}
\usepackage[dvipsnames]{xcolor}
\usepackage{geometry}
\usepackage{graphicx}
\usepackage{epstopdf}
\epstopdfsetup{update}
\geometry{margin= 2cm}
\usepackage[makeroom]{cancel}
\usepackage{multirow}
\setlength\parindent{0pt}


\begin{document}
	\pagenumbering{gobble}
	\section*{8. Numerik Übungen 2017/18}
	\paragraph{T14}\mbox{}\\
	\textbf{%
		Berechnen Sie das Integral
		\begin{align*}
			\mathbf{\int\limits_{0}^{2}\frac{1}{1+x}\text{dx}}
		\end{align*}
		mit der Trapezregel und schätzen Sie den Fehler mit Hilfe der Richardson-Extrapolation, d.h. bestimmen Sie für $\mathbf{h=2}$ die Werte $\mathbf{A(h)}$, $\mathbf{A(h/2)}$ und est. Berechnen Sie auch $\mathbf{Q_\text{extr.}(f)}$.
	}\\
	Mit einem Auszug aus der Definition der Quadraturformel
	\begin{align*}\tag{(4.12)}
		I(g)=\int\limits_{0}^{1}g(t)\text{dt} \approx \sum\limits_{\text{i}=1}^{s}b_ig(c_i)=:Q(g)
	\end{align*}
	folgt analog zu Bsp. 4.2 (S. 79) mit der Trapetzregel: $s = 2$, $c_i=[0 \quad 1]$, $b_i = [\frac{1}{2} \quad \frac{1}{2}]$
	\begin{align*}
		\int\limits_{0}^{2}\frac{1}{1+x}\text{dx} &\approx \sum\limits_{\text{i}=1}^{s} b_i \frac{1}{1+c_i}\\
		 & \approx b_1 \frac{1}{1+c_1}+ b_2 \frac{1}{1+c_2} \\
		 & \approx \frac{1}{2} \left( \frac{1}{1+0}+ \frac{1}{1+1} \right) \\
		 & \approx \frac{3}{4}.
	\end{align*}
	
	
	Aus der allgemeinen Formel für die Approximation
	\begin{align*}\tag{4.85, S. 99}
		A(h):=Q(f,[\alpha,\alpha+h])=h\sum\limits_{\text{i}=1}^{s}b_if(\alpha+c_ih)\approx\int\limits_{\alpha}^{\alpha+h}f(x)\text{dx}
	\end{align*}
	folgt wiederum mit der Trapetzregel
	\begin{align*}
		A(h)&=h\,\Bigl( b_1g(0+c_1h) + b_2g(0+c_2h)\Bigr) \\
		&=2\,\Bigl( \frac{1}{2}g(0+0\cdot 2) + \frac{1}{2}g(0+1\cdot 2)\Bigr) \\
		&=g(0) + g(2) \\
		&=\frac{1}{1+0} + \frac{1}{1+2} \\
		&=\frac{4}{3}.
	\end{align*}
	Die im Allgemeinen bessere Approximation A(h/2) mit Schrittweite h/2 lautet
	\begin{align*}
		A(h/2):&=Q(f,[\alpha,\alpha+h/2])+Q(f,[\alpha+h/2,\alpha+h]) \tag{4.86} \\
		&=\frac{h}{2}\sum\limits_{\text{i}=1}^{s}b_if\Bigl(\alpha+c_i\frac{h}{2}\Bigr) + \frac{h}{2}\sum\limits_{\text{i}=1}^{s}b_if\Bigl(\alpha+\frac{h}{2}+c_i\frac{h}{2}\Bigr) \\
		&= \frac{h}{2}\,\Biggl(
				\sum\limits_{\text{i}=1}^{s}b_ig\Bigl(\alpha+c_i\frac{h}{2}\Bigr) + \sum\limits_{\text{i}=1}^{s}b_ig\Bigl(\alpha+\frac{h}{2}+c_i\frac{h}{2}\Bigr)
		   \Biggr) \\
		&= \frac{h}{2}\,\left(
			\sum\limits_{\text{i}=1}^{s}b_ig\Bigl(0+c_i\frac{h}{2}\Bigr) +  \sum\limits_{\text{i}=1}^{s}b_ig\Bigl(0+\frac{h}{2}+c_i\frac{h}{2}\Bigr)
		   \right) \\
		&= \frac{h}{2}\,\left(
			\left[b_1g\Bigl(c_1\frac{h}{2}\Bigr) + b_2g\Bigl(c_2\frac{h}{2}\Bigr) \right]+  	\left[b_1g\Bigl(\frac{h}{2}(1+c_1)\Bigr) +b_2g\Bigl(\frac{h}{2}(1+c_2)\Bigr) \right]
		   \right) \\
		&= \frac{2}{2}\,\left(
			\left[\frac{1}{2}g\Bigl(0\cdot\frac{2}{2}\Bigr) + \frac{1}{2}g\Bigl(1\cdot\frac{2}{2}\Bigr) \right]+\left[\frac{1}{2}g\Bigl(\frac{2}{2}(1+0)\Bigr) +\frac{1}{2}g\Bigl(\frac{2}{2}(1+1)\Bigr) \right]
		   \right) \\
		&= \frac{1}{2}\,\Biggl( \Bigl[g(0) + g(1)\Bigr] + \Bigl[g(1) + g(2)\Bigr] \Biggr) \\
		&= \frac{1}{2}\,\Biggl( \Bigl[\frac{1}{1+0} + \frac{1}{1+1}\Bigr] + \Bigl[\frac{1}{1+1} + \frac{1}{1+2}\Bigr] \Biggr) \\
		&= \frac{1}{2}\,\Biggl( \Bigl[1 + \frac{1}{2}\Bigr] + \Bigl[\frac{1}{2} + \frac{1}{3}\Bigr] \Biggr) \\
		&= \frac{1}{2}\,\Biggl(2 + \frac{1}{3} \Biggr) \\
		&= \frac{7}{6}.\\
	\end{align*}
	Damit lässt sich die Schätzung est für den Fehler von A(h/2) ermitteln, wobei Ordnung $p=2$ aus Bsp. 4.4, S. 85 bekannt ist.
	\begin{align*}
		\int\limits_{\alpha}^{\alpha+h}f(x)\text{dx}-A(h/2)\approx \frac{A(h/2)-A(h)}{2^p-1}&=:\text{est} \tag{4.92, S. 100} \\
		\text{est}&=\frac{\frac{7}{6}-\frac{4}{3}}{2^2-1} \\
		&=\frac{-\frac{1}{3}}{3} \\
		&=-\frac{1}{9} = 1.\dot{1}
	\end{align*}
	Mit der Richardson-Extrapolation erhält man einen verbesserten Wert, auch extrapolierten Wert, für das bestimmte Integral
	\begin{align*}
		\int\limits_{\alpha}^{\alpha+h}f(x)\,\text{dx}\approx A(h/2)+\text{est}&=:Q_{\text{extr.}}(f) \\
		Q_{\text{extr.}}(f)&=\frac{7}{6}+\left(-\frac{1}{9} \right) \\
		&=\frac{21-2}{18} \\
		&=\frac{19}{18}=1.0\dot{5} \\
	\end{align*}
	Die exakte Lösung ist $\log(3)\approx1.0986$, die verbleibende Differenz beträgt $0.043$ ($\Delta3.9\%$).
\end{document}