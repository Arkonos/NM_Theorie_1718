% !TeX spellcheck = de_AT_frami
\documentclass[10pt,a4paper]{article}
\usepackage[utf8x]{inputenc}
\usepackage{ucs}
\usepackage{amsmath}
\usepackage{setspace}
\usepackage{amsfonts}
\usepackage{amssymb}
\usepackage{graphicx}
\usepackage{txfonts}
\usepackage[dvipsnames]{xcolor}
\usepackage{geometry}
\usepackage{graphicx}
\usepackage{epstopdf}
\epstopdfsetup{update}
\geometry{margin= 2cm}
\usepackage[makeroom]{cancel}
\usepackage{multirow}
\setlength\parindent{0pt}

\usepackage{scalerel,stackengine}
\stackMath
\newcommand\reallywidehat[1]{%
	\savestack{\tmpbox}{\stretchto{%
			\scaleto{%
				\scalerel*[\widthof{\ensuremath{#1}}]{\kern-.6pt\bigwedge\kern-.6pt}%
				{\rule[-\textheight/2]{1ex}{\textheight}}%WIDTH-LIMITED BIG WEDGE
			}{\textheight}% 
		}{0.5ex}}%
	\stackon[1pt]{#1}{\tmpbox}%
}
\parskip 1ex


\begin{document}
	\pagenumbering{gobble}
	\section*{7. Numerik Übungen 2017/18}
	\paragraph{T12}\mbox{}\\
	\textbf{%
		a) Stellen Sie die Bedingungsgleichung für die Simpsonregel auf und bestimmen Sie damit aus den Knoten $\pmb{c_1=0}$, $\pmb{c_2=\frac{1}{2}}$, und $\pmb{c_3=1}$ die Gewichte.\\
        Welche Ordnung besitzt die Simpsonregel? Untersuchen Sie dazu, ob eventuell noch weitere Bedingungsgleichungen erfüllt sind.
	}\\
	
    
    
    \textbf{%
        b) Gegeben seien die Knoten $\pmb{c_1=\frac{1}{6}}$, $\pmb{c_2=\frac{1}{2}}$, und $\pmb{c_3=\frac{5}{6}}$. Stellen Sie die ersten $\pmb{s}$ Bedingungsgleichungen auf und setzten Sie die Knoten $\pmb{c_1, c_2, c_3}$ ein. Berechnen Sie daraus die Gewichte. Wie groß ist die Ordnung dieser Quadraturformel?
    }\\


    \textbf{%
        c) Bestimmen Sie alternativ die Gewichte $\pmb{b_1, b_2, b_3}$ durch Integration der zu den Knoten $\pmb{c_1, c_2, c_3}$ gehörigen Lagrange-Polynome $\pmb{l_1, l_2, l_3}$.
    }\\


    \textbf{%
        d) Welche Ordnung hat eine Quadraturformel mit Knoten wie in (T12b) und Gewichten $\pmb{b_1=\frac{1}{3}, b_2=\frac{1}{3}, b_3=\frac{1}{3}}$?
    }\\



    \pagebreak
    \paragraph{T13}\mbox{}\\
    \textbf{%
        Berechnen Sie das Integral
        \begin{align*}
            \int_{-1}^{2}\frac{1}{2+x}dx.
        \end{align*}
        a) Exakt.
        b) Mit der Quadraturformel aus Aufgabe (T12b) und Schrittweite $h=3$. \\
        c) Mit der Quadraturformel aus Aufgabe (T12b) und Schrittweite $h=\frac{3}{2}$. \\
        Machen Sie eine Skizze mit den Knoten und Gewichten.
    }\\
	
\end{document}