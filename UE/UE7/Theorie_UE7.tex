% !TeX spellcheck = de_AT_frami
\documentclass[10pt,a4paper]{article}
\usepackage[utf8x]{inputenc}
\usepackage{ucs}
\usepackage{amsmath}
\usepackage{setspace}
\usepackage{amsfonts}
\usepackage{amssymb}
\usepackage{graphicx}
\usepackage{txfonts}
\usepackage[dvipsnames]{xcolor}
\usepackage{geometry}
\usepackage{graphicx}
\usepackage{epstopdf}
\epstopdfsetup{update}
\geometry{margin= 2cm}
\usepackage[makeroom]{cancel}
\usepackage{multirow}
\setlength\parindent{0pt}

\usepackage{scalerel,stackengine}
\stackMath
\newcommand\reallywidehat[1]{%
	\savestack{\tmpbox}{\stretchto{%
			\scaleto{%
				\scalerel*[\widthof{\ensuremath{#1}}]{\kern-.6pt\bigwedge\kern-.6pt}%
				{\rule[-\textheight/2]{1ex}{\textheight}}%WIDTH-LIMITED BIG WEDGE
			}{\textheight}% 
		}{0.5ex}}%
	\stackon[1pt]{#1}{\tmpbox}%
}
\parskip 1ex


\begin{document}
	\pagenumbering{gobble}
	\section*{7. Numerik Übungen 2017/18}
	\paragraph{T12}\mbox{}\\
	\textbf{%
		a) Stellen Sie die Bedingungsgleichung für die Simpsonregel auf und bestimmen Sie damit aus den Knoten $\pmb{c_1=0}$, $\pmb{c_2=\frac{1}{2}}$, und $\pmb{c_3=1}$ die Gewichte.\\
        Welche Ordnung besitzt die Simpsonregel? Untersuchen Sie dazu, ob eventuell noch weitere Bedingungsgleichungen erfüllt sind.
	}\\
    Bedingungsgleichung allgemein (S. 84)
	\begin{align}\tag{4.31}
		\sum_{i=1}^{s}b_ic_i^k = \frac{1}{k+1}, k=0, \dots, p-1
	\end{align}
	Die Simpsonregel hat 3 Knoten $c_i$, und somit mindestens Ordnung $p = 3$.
	\begin{alignat*}{5}
		k=0 & \quad & b_1\cdot 0^0+ & b_2 \, & \left(\frac{1}{2}\right)^0+ & b_3\cdot1^0 & = \frac{1}{0+1} \\
		k=1 & \quad & b_1\cdot 0^1+ & b_2 \, & \left(\frac{1}{2}\right)^1+ & b_3\cdot1^1 & = \frac{1}{1+1} \\
		k=2 & \quad & b_1\cdot 0^2+ & b_2 \, & \left(\frac{1}{2}\right)^2+ & b_3\cdot1^2 & = \frac{1}{2+1}
		\end{alignat*}
		Mit $k=0$ und $i=0$ bekommen wir $0^0$, was normalerweise nicht definiert ist. Das können wir in diesem Fall durch Wahl von Grenzwerten 1 setzen, vergleichbar mit $\frac{x}{0}=\infty$. (Quelle: Mathematik Masterstudent)
		\begin{align*}
			\begingroup % keep the change local
			\setstretch{1.3}
			\begin{bmatrix}
				1 & 1 & 1 \\
				0 & \frac{1}{2} & 1 \\
				0 & \frac{1}{4} & 1 \\
			\end{bmatrix}
			\endgroup
			\begingroup % keep the change local
			\setstretch{1.3}
				\begin{bmatrix}
					b_1 \\
					b_2 \\
					b_3
				\end{bmatrix}
				\endgroup
				=
				\begingroup % keep the change local
				\setstretch{1.3}
				\begin{bmatrix}
					1 \\
					\frac{1}{2} \\
					\frac{1}{3} \\
				\end{bmatrix}
				\endgroup \Rightarrow \begingroup % keep the change local
				\setstretch{1.3}\begin{bmatrix}
				b_1 \\
				b_2 \\
				b_3
				\end{bmatrix}\endgroup =
				\begingroup % keep the change local 
				\setstretch{1.3}
				\begin{bmatrix}
				\frac{1}{6} \\
				\frac{2}{3} \\
				\frac{1}{6}
				\end{bmatrix}\endgroup
		\end{align*}
		\begin{align*}
			k=3  \quad  b_1\cdot 0^3+        b_2         \,  \left(\frac{1}{2} \right)^3+  b_3\cdot1^3 & = \frac{1}{3+1} \\
			\frac{1}{6}\cdot 0+  \frac{2}{3}\cdot \frac{1}{8}  +  \frac{1}{6}\cdot 1                   & = \frac{1}{4}   \\
			\frac{1}{3}\cdot \frac{1}{4}  +  \frac{1}{6}\cdot 1                                        & = \frac{1}{4}   \\
			\frac{1}{12} +  \frac{2}{12}                                                               & = \frac{1}{4}   \\
			\frac{3}{12}                                                                               & = \frac{1}{4}   \\
			\frac{1}{4}                                                                                & = \frac{1}{4}   \\
			k=4  \quad  b_1\cdot 0^4+        b_2         \,  \left(\frac{1}{2} \right)^4+  b_3\cdot1^4 & = \frac{1}{4+1} \\
			\frac{1}{6}\cdot 0+  \frac{2}{3}\cdot \frac{1}{16}  +  \frac{1}{6}\cdot 1                  & = \frac{1}{5}   \\
			\frac{2}{48} +  \frac{1}{6}                                                                & = \frac{1}{5}   \\
			\frac{1}{24} +  \frac{4}{24}                                                               & = \frac{1}{5}   \\
			\frac{5}{24}                                                                               & = \frac{1}{5}
		\end{align*}
		Wie man sieht Integriert die Simpsonregel bis zum Grad 4 genau.
    
    \pagebreak
    \textbf{%
        b) Gegeben seien die Knoten $\pmb{c_1=\frac{1}{6}}$, $\pmb{c_2=\frac{1}{2}}$, und $\pmb{c_3=\frac{5}{6}}$. Stellen Sie die ersten $\pmb{s}$ Bedingungsgleichungen auf und setzten Sie die Knoten $\pmb{c_1, c_2, c_3}$ ein. Berechnen Sie daraus die Gewichte. Wie groß ist die Ordnung dieser Quadraturformel?
    }\\
		Analog zu a), aber diesmal direkt mit der Matrixschreibweise aus (4.34).
		\begin{align*}
			\begingroup % keep the change local
				\setstretch{1.3}
				\begin{bmatrix}
					1         & 1        & \dots & 1        \\
					c_1       & c_2      & \dots & c_s      \\
					\vdots    & \vdots   & \dots & \vdots   \\
					c_1^{p-1} & c_2{p-1} & \dots & c_s{p-1}
				\end{bmatrix}
			\endgroup
			\begingroup % keep the change local
				\setstretch{1.3}
				\begin{bmatrix}
					b_1 \\
					b_2 \\
					\vdots \\
					b_s
				\end{bmatrix}
			\endgroup
			=
			\begingroup % keep the change local
				\setstretch{1.3}
				\begin{bmatrix}
					1 \\
					\frac{1}{2} \\
					\vdots \\
					\frac{1}{p} \\
				\end{bmatrix}
			\endgroup \\
			\begingroup % keep the change local
				\setstretch{1.3}
				\begin{bmatrix}
					1 & 1 & 1 \\
					\frac{1}{6} & \frac{1}{2} & \frac{5}{6} \\
					\frac{1}{12} & \frac{1}{4} & \frac{25}{36} \\
				\end{bmatrix}
			\endgroup
			\begingroup % keep the change local
				\setstretch{1.3}
				\begin{bmatrix}
					b_1 \\
					b_2 \\
					b_3
				\end{bmatrix}
			\endgroup
			=
			\begingroup % keep the change local
				\setstretch{1.3}
				\begin{bmatrix}
					1 \\
					\frac{1}{2} \\
					\frac{1}{3} \\
				\end{bmatrix}
			\endgroup \Rightarrow 
				\begingroup % keep the change local
					\setstretch{1.3}\begin{bmatrix}
						b_1 \\
						b_2 \\
						b_3
					\end{bmatrix}\endgroup =
				\begingroup % keep the change local 
					\setstretch{1.3}
					\begin{bmatrix}
						\frac{3}{10} \\
						\frac{2}{5} \\
						\frac{3}{10} \\
					\end{bmatrix}
				\endgroup
		\end{align*}
		\begin{align*}
			k=3  \quad  b_1\cdot \left( \frac{1}{6}\right)^3+  b_2  \, \left(\frac{1}{2} \right)^3+  b_3\cdot\left( \frac{5}{6}\right)^3 & = \frac{1}{3+1} \\
			\vdots \,\,\, & = \frac{1}{4} \\
			\frac{9}{40} &= \frac{1}{4}
		\end{align*}
		Alternativ kann man auch eine $k+1$te Zeile auf lineare Abhängigkeit in der Matrizen-Schreibweise überprüfen. \\
		Die gegebene Quadraturformel ist bis zum Grad 3 genau.
		
		\newpage
		
    \textbf{%
        c) Bestimmen Sie alternativ die Gewichte $\pmb{b_1, b_2, b_3}$ durch Integration der zu den Knoten $\pmb{c_1, c_2, c_3}$ gehörigen Lagrange-Polynome $\pmb{l_1, l_2, l_3}$.
    }\\


    \textbf{%
        d) Welche Ordnung hat eine Quadraturformel mit Knoten wie in (T12b) und Gewichten $\pmb{b_1=\frac{1}{3}, b_2=\frac{1}{3}, b_3=\frac{1}{3}}$?
    }\\
	\begin{align*}
		c = \begingroup % keep the change local
			\setstretch{1.3}
			\begin{bmatrix}
				\frac{1}{6} \\
				\frac{1}{2} \\
				\frac{5}{6}
			\end{bmatrix}
			\endgroup,  \quad
		b = \begingroup % keep the change local
			\setstretch{1.3}
			\begin{bmatrix}
				\frac{1}{3} \\
				\frac{1}{3} \\
				\frac{1}{3}
			\end{bmatrix}
			\endgroup
	\end{align*}
	\begin{align*}
		k=0 \hspace{4.5cm}1&=1 \\
		k=1  \quad  b_1\cdot \left( \frac{1}{6}\right)^1+  b_2  \, \left(\frac{1}{2} \right)^1+  b_3\cdot\left( \frac{5}{6}\right)^1 & = \frac{1}{1+1} \\
		\frac{1}{3}\frac{1}{6}+ \frac{1}{3}  \frac{1}{2} + \frac{1}{3} \frac{5}{6}                                                   & = \frac{1}{2} \\
		\frac{1}{2} & = \frac{1}{2} \\
		k=2 \hspace{0.1cm} \quad  \frac{1}{3}\cdot \left( \frac{1}{6}\right)^2+  \frac{1}{3}  \, \left(\frac{1}{2} \right)^2+  \frac{1}{3}\cdot\left( \frac{5}{6}\right)^2 & = \frac{1}{2+1} \\
		\frac{1}{3}\frac{1}{36}+  \frac{1}{3}  \frac{1}{4} + \frac{1}{3} \frac{25}{36} & = \frac{1}{3} \\
		\frac{35}{18} &= \frac{1}{3}
	\end{align*}
	Die gegebene Quadraturformel die Ordnung 3.
    \pagebreak
    \paragraph{T13}\mbox{}\\
    \textbf{%
        Berechnen Sie das Integral
        \begin{align*}
            \int_{-1}^{2}\frac{1}{2+x}dx.
        \end{align*}
        a) Exakt.
        b) Mit der Quadraturformel aus Aufgabe (T12b) und Schrittweite $h=3$. \\
        c) Mit der Quadraturformel aus Aufgabe (T12b) und Schrittweite $h=\frac{3}{2}$. \\
        Machen Sie eine Skizze mit den Knoten und Gewichten.
    }\\
	
\end{document}