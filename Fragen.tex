% !TeX spellcheck = de_AT_frami
\documentclass[10pt,a4paper]{article}
\usepackage[utf8x]{inputenc}
\usepackage{ucs}
\usepackage{amsmath}
\usepackage{amsfonts}
\usepackage{amssymb}
\usepackage{bm}
\usepackage{graphicx}
\usepackage{txfonts}
\usepackage[dvipsnames]{xcolor}
\usepackage{geometry}
\usepackage{graphicx}
\usepackage{epstopdf}
\epstopdfsetup{update}
\geometry{margin= 2cm}

\setlength\parindent{0pt}
\renewcommand*{\theenumi}{\thesection.\arabic{enumi}}
\renewcommand*{\theenumii}{\theenumi.\arabic{enumii}}

\begin{document}
	\pagenumbering{gobble}
	\title{Numerische Mathematik
		\\
		2017/18
		\\
		\flushbottom
		Theoriefragen }
	\maketitle
	
	\newpage
	\section{Zahldarstellung, Rundung und Fehler}
	\begin{enumerate}
		\item \textbf{Wie werden ganze Zahlen binär abgespeichert?}\\
			wat
		\item \textbf{Wie werden Gleitpunktzahlen (doppelte Genauigkeit) binär abgespeichert?} \\
		\item \textbf{Wie werden Gleitpunktzahlen gerundet?} \\
		\item \textbf{Wie ist der relative Rundungsfehler definiert?} \\
		\item \textbf{Wie groß ist die relative Maschinengenauigkeit \texttt{eps} für doppelt genaue Gleitpunktzahlen?\\
					Wie kann man eps experimentell bestimmen?} \\
		\item \textbf{Was ist die relative/absolute Kondition eines Problems?} \\
		\item \textbf{Was bedeuten die Begriffe Konsistenz und Konsistenzordnung?} \\
		\item \textbf{Wodurch unterscheidet sich Konsistenz von Konvergenz?} \\
		\item \textbf{Was bedeutet der Begriff Stabilität?}
	\end{enumerate}
	
	\newpage
	\section{Numerische Differentiation}
	\begin{enumerate}
		\item \textbf{Wie wird mit Hilfe der Vorwärtsdifferenz eine differenzierbare Funktion $\pmb{f}$ an der Stelle $\pmb{x}$ differenziert? Wie groß ist $\pmb{h_opt}$?}\\
		\item \textbf{Wie wird mit Hilfe der zentralen Differenz eine differenzierbare Funktion $\pmb{f}$ an der Stelle $\pmb{x}$ differenziert? Wie groß ist $\pmb{h_opt}$}\\
		\item \textbf{Wie verhalten sich Verfahrensfehler und Rundungsfehler in Abhängigkeit von der Schrittweite $\pmb{h}$? Machen Sie eine Skizze.}\\
		\item \textbf{Wie lässt sich mit Hilfe eines logarithmischen Plots das Verhalten von Verfahrensfehler und Rundungsfehler ablesen? Wie kann man die optimale Schrittweite $\pmb{h_opt}$ ablesen?}\\
		\item \textbf{Wieso gilt bei der zentralen Differenz für den Verfahrensfehler $\pmb{V(h)=\mathcal{O}(h^2)}$ statt $\pmb{\mathcal{O}(h)}$?}\\
		\item \textbf{Wie wird die zweite Ableitung einer zweimal differenzierbaren Funktion an der Stelle $\pmb{x}$ berechnet? Wie groß ist $\pmb{h_opt}$?}\\
		\item \textbf{Wie lässt sich die optimale Schrittweite $\pmb{h_opt}$ aus dem Verfahrensfehler $\pmb{V(h)}$ und dem Rundungsfehler $\pmb{R(h)}$ bestimmen?}\\
		\item \textbf{Wie berechnet man die Jacobimatrix einer vektorwertigen Funktion $\pmb{f:}{\rm I\!R}\pmb{^n} \rightarrow {\rm I\!R}\pmb{^m}$ durch numerisches Differenzieren?}\\
		\item \textbf{Wie berechnet man den Gradient einer skalaren Funktion $\pmb{f:}{\rm I\!R}\pmb{^n} \rightarrow {\rm I\!R}\pmb{^m}$ durch numerisches Differenzieren?}\\
	\end{enumerate}

	\newpage
	\section{Interpolation}
	\begin{enumerate}
		\item \textbf{Wie werden die dividierten Differenzen berechnet?} \\
		
		\item \textbf{Wie ist das Newtonsche Interpolationspolynom definiert?} \\
		
		\item \textbf{Wie wird mit dem Hornerschema ein Polynom $\pmb{p(x)=a_0+a_1x+\dots +a_nx^n}$ ausgewertet?} \\
		
		\item \textbf{Wie wird mit dem Hornerschema ein Newtonsches Interpolationspolynom ausgewertet?} \\
		
		\item \textbf{Wie sind die Lagrange-Polynome definiert? Welche Eigenschaften haben sie?} \\
		
		\item \textbf{Wie wird mit Hilfe der Lagrange-Polynome das Langrangesche Interpolationspolynom berechnet?} \\
		
		\item \textbf{Erklären Sie die Begriffe Datenfehler, Verstärkungsfaktor, Lebesgue-Funktion und Lebesgue-Konstante in Zusammenhang mit der Polynominterpolation. Was ist die Kondition der Polynominterpolation?} \\
		
		\item \textbf{Was besagt der Satz über den Fehler des Interpolationspolynoms? Wie ist der Verfahrensfehler definiert?} \\
		
		\item \textbf{Wie sind die Tschebyscheff-Polynome definiert? Welche Eigenschaften haben sie?} \\
		
		\item \textbf{Wie berechnet man die Knoten für die Tschebyscheff-Interpolationspolynome im Intervall [−1, 1] bzw. [a, b]? Welche Vorteile hat die Verwendung von Tschebyscheff-Knoten im Vergleich zu äquidistanten Stützstellen.} \\
		
		\item \textbf{Wie lässt sich das dividierte Differenzenschema und das Newtonsche Interpolationspolynom verallgemeinern, falls in den Stützstellen auch noch Ableitungen vorgegeben sind?} \\
		
		\item \textbf{Wie wird mit stückweise konstanten Funktionen interpoliert?} \\
		
		\item \textbf{Wie wird mit stetigen, stückweise linearen Funktionen interpoliert?} \\
		
		\item \textbf{Was sind Hutfunktionen und welche Eigenschaften haben sie?} \\
		
		\item \textbf{Was für Eigenschaften besitzen kubische Splines? Was für Typen von kubischen Splines gibt es?} \\
		
		\item \textbf{Wieso ist es besser durch viele Punkte einen kubischen Spline zu legen, statt ein Interpolationspolynom zu verwenden?} \\
		
		\item \textbf{Wie wird auf einem rechteckigen Gitter zweidimensional interpoliert?} \\
		
		\item \textbf{Wie wird die zweidimensionale, stetige, stückweise lineare Interpolierende auf einem rechteckigen Gitter bestimmt?} \\
		
	\end{enumerate}

	\newpage
	\section{Numerische Integration}
	\begin{enumerate}
		\item \textbf{Was bedeutet \textit{Linearität} und \textit{Positivität} des Integrals?} \\
		
		\item \textbf{Erklären Sie den Begriff \textit{Quadraturformel}.} \\
		
		\item \textbf{Nennen Sie einige einfache Quadraturformeln inklusive Knoten und Gewichte.} \\
		
		\item \textbf{Erklären Sie den Begriff \textit{zusammengesetzte Quadraturformel}.} \\
		
		\item \textbf{Wie erhält man Quadraturformeln mit Hilfe von Polynominterpolation?} \\
		
		\item \textbf{Erklären Sie den Begriff \textit{Ordnung} einer Quadraturformel. Wie bestimmt man die Ordnung?} \\
		
		\item \textbf{Was sind \textit{Bedingungsgleichungen}?} \\
		
		\item \textbf{Erklären Sie die Begriffe \textit{Fehler einer Quadraturformel} und \textit{Fehlerkonstante}.} \\
		
		\item \textbf{Was für Abschätzungen gelten für den Fehler einer Quadraturformel bzw. einer zusammengesetzten Quadraturformel? Was muss der Integrand $f$ dabei erfüllen?} \\
		
		\item \textbf{Was sind \textit{symmetrische Quadraturformeln} und welche Eigenschaft besitzen sie?} \\
		
		\item \textbf{Was ist eine Gaußsche Quadraturformel? Welche Ordnung besitzen sie?} \\
		
		\item \textbf{Wie groß kann die Ordnung einer Quadraturformel maximal sein?} \\
		
		\item \textbf{Was gilt für die Gewichte einer Gaußschen Quadraturformel?} \\
		
		\item \textbf{Wie funktioniert eine Schrittweitensteuerung? Erklären Sie die Begriffe \textit{Fehlerkriterium} und \textit{Fehlerschätzer}.} \\
		
		\item \textbf{Erklären Sie den Begriff \textit{Richardson-Extrapolation}. Wie berechnet man est und $\pmb{Q_extr}$.} \\
		
		\item \textbf{Was passiert bei Integranden mit Singularitäten oder Singularitäten in den Ableitungen?} \\
		
		\item \textbf{Wie werden Doppelintegrale auf Rechtecken numerisch berechnet?} \\
		
		\item \textbf{Wie werden Doppelintegrale auf Dreiecken numerisch berechnet? Wie überprüft man die Ordnung einer Quadraturformel für Dreiecke?} \\
		
		\item \textbf{Was sind \textit{baryzentrische Koordinaten}?} \\
		
	\end{enumerate}
\end{document}