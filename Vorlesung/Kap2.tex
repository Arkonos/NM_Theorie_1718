\section{Numerische Differentiation}
	\begin{enumerate}
		\item \textbf{Wie wird mit Hilfe der Vorwärtsdifferenz eine differenzierbare Funktion $\pmb{f}$ an der Stelle $\pmb{x}$ differenziert? Wie groß ist $\pmb{h_{opt}}$?}\\
		
		\item \textbf{Wie wird mit Hilfe der zentralen Differenz eine differenzierbare Funktion $\pmb{f}$ an der Stelle $\pmb{x}$ differenziert? Wie groß ist $\pmb{h_{opt}}$}\\
		\item \textbf{Wie verhalten sich Verfahrensfehler und Rundungsfehler in Abhängigkeit von der Schrittweite $\pmb{h}$? Machen Sie eine Skizze.}\\
		\item \textbf{Wie lässt sich mit Hilfe eines logarithmischen Plots das Verhalten von Verfahrensfehler und Rundungsfehler ablesen? Wie kann man die optimale Schrittweite $\pmb{h_{opt}}$ ablesen?}\\
		\item \textbf{Wieso gilt bei der zentralen Differenz für den Verfahrensfehler $\pmb{V(h)=\mathcal{O}(h^2)}$ statt $\pmb{\mathcal{O}(h)}$?}\\
		\item \textbf{Wie wird die zweite Ableitung einer zweimal differenzierbaren Funktion an der Stelle $\pmb{x}$ berechnet? Wie groß ist $\pmb{h_{opt}}$?}\\
		\item \textbf{Wie lässt sich die optimale Schrittweite $\pmb{h_{opt}}$ aus dem Verfahrensfehler $\pmb{V(h)}$ und dem Rundungsfehler $\pmb{R(h)}$ bestimmen?}\\
		\item \textbf{Wie berechnet man die Jacobimatrix einer vektorwertigen Funktion $\pmb{f:}{\rm I\!R}\pmb{^n} \rightarrow {\rm I\!R}\pmb{^m}$ durch numerisches Differenzieren?}\\
		\item \textbf{Wie berechnet man den Gradient einer skalaren Funktion $\pmb{f:}{\rm I\!R}\pmb{^n} \rightarrow {\rm I\!R}\pmb{^m}$ durch numerisches Differenzieren?}\\
	\end{enumerate}