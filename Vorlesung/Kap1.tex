% !TeX spellcheck = de_AT_frami
\section{Zahldarstellung, Rundung und Fehler}
	\begin{enumerate}
		\item \textbf{Wie werden ganze Zahlen binär abgespeichert?}\\
			S. 2
			\begin{align*}
				b_{N-1}b_{N-2}\dots b_{1}b_{0} \cong b = \sum_{j=0}^{N-1}b_j2^j, \quad b_j\in\{0,1\}
			\end{align*}
			Beispiel: \(23_{10}\)
			\begin{align*}
					10101_2 &= 1\cdot2^4+0\cdot2^3+0\cdot2^2+1\cdot2^1+1\cdot2^0 \\
					      &= 16+0+0+2+1 = 19_{10}
			\end{align*} \vspace{-1cm}
			\begin{figure}[htbp]
				\centering
				\begin{minipage}{0.3\textwidth}
					\centering
					$$\begin{array}{rrrrrr}
					\multirow{2}{*}{:2} & 19 & 9 & 4 & 2 & 1 \\
					\cline{2-6}
					 \rule{0pt}{2.5ex} & 1 &  1 & 0 & 0 & 1
					\end{array}$$
				\end{minipage}\hspace{-0.5cm}
				\begin{minipage}{0.01\textwidth}
					\centering
					\vspace{0.3cm}
					\(\rightarrow\)
				\end{minipage}\hspace{-0.7cm}
				\begin{minipage}{0.2\textwidth}
					\centering
					\vspace{0.3cm}
					\(10011_2\)
				\end{minipage}
			\end{figure}
			
			
		\item \textbf{Wie werden Gleitpunktzahlen (doppelte Genauigkeit) binär abgespeichert?}\\
		S. 3
			\begin{align*}
				x &= (-1)^s \cdot m \cdot 2^e \\
				x &\cong s \;\;\; e_{11}e_{10}\dots e_0 \;\;\; (m_0).m_1m_2\dots m_{51}
			\end{align*}\vspace{-0.5cm}
			\begin{table}[htbp]
				\centering
				\begin{tabular}[htpb]{rll}
					s \; \dots \!\!\! & Vorzeichenbit &\(\in \{0,1\}\)\\
					m \; \dots \!\!\! & Mantisse & Normiert, \(m_0\overset{!}{=}1\) wird weggelassen\\
					e \; \dots \!\!\! & Exponent & nach Abzug von b\(\dots\)Bias = 1023 (double)
				\end{tabular} \\ \vspace{0.3cm}
				\begin{tabular}[htpb]{rccc}
					 & s & m & l \\
					  single 32 Bit & 1 & 23 & 8 \\
					  double 64 Bit & 1 & 52 & 11
				\end{tabular}
			\end{table}
		\item \textbf{Wie werden Gleitpunktzahlen gerundet?} \\
		S. 6 \\
		Round to the nearest even.
		\begin{table}[htbp]
			\centering
			\begin{tabular}[htpb]{ccccccl}
				\dots & \(\text{m}_\text{M}\) & \(\text{m}_{\text{M}+1}\) & \(\text{m}_{\text{M}+2}\) & \(m_{M+3}\) & \dots &  \\
				      & x                   & 0                       & x                       & x         &       & abrunden  \\
				      & x                   & 1                       & 1                       & 0         &       & aufrunden \\
				      & x                   & 1                       & 0                       & 1         &       & aufrunden \\
				      & 1                   & 1                       & 0                       & 0         &       & aufrunden \\
				      & 0                   & 1                       & 0                       & 0         &       & abrunden
			\end{tabular}
		\end{table}
		\item \textbf{Wie ist der relative Rundungsfehler definiert?} \\
		S. 6
		\begin{align*}
			\frac{\abs{\text{rd}(x)-x}}{\abs{x}}\leq \frac{2^{-\text{M}-1}\cdot2^e}{\text{a}\cdot 2^e}\underset{\text{a}\in[1,2)}{\leq}2^{-\text{M}-1}=:\texttt{eps} \\
			\text{rd}(a) \text{ durch rounding to the nearest even} 
		\end{align*}
		
		\item \textbf{Wie groß ist die relative Maschinengenauigkeit \texttt{eps} für doppelt genaue Gleitpunktzahlen?\\
					Wie kann man \texttt{eps} experimentell bestimmen?} \\
				
				
		\item \textbf{Was ist die relative/absolute Kondition eines Problems?}
			\begin{align*}
				\frac{\abs{f(\tilde{x})-f(x)}}{\abs{f(x)}} &\leq \kappa_{\text{rel.}}\cdot\varepsilon, \quad \kappa_{\text{rel.}} > 0 \\
				\abs{f(\tilde{x})-f(x)} &\leq \kappa_{\text{abs.}}\cdot\delta, \quad \kappa_{\text{abs.}} > 0
			\end{align*}
		Wenn \(\kappa_{\text{rel.}}\) klein ist werden Inputfehler nicht übermäßig verstärkt und \(f\) gilt als gut konditioniert.
		\item \textbf{Was bedeuten die Begriffe Konsistenz und Konsistenzordnung?} \\
			Def 1.3 Konsistenz, Konsistenzordnung \\
			Ein numerisches Verfahren \(f_h\) mit Diskretisierungsweite h zur Bestimmung einer Näherung von \(f_h(x)\) an \(f(x)\) ist konsistent, falls gilt
			\begin{align*}
				\norm{f_h(x)-f(x)}\leq \text{C}\,h^\text{p}
			\end{align*}
			wobei die Konstante \(\text{C}>0\) nicht von h abhängen darf und \(p\ge1\). Der Exponent \(p\) ist dann die Konsistenzordnung und es gilt \(f_h\rightarrow f\) für \( h \rightarrow 0\), falls \(f_h\) exakt ausgewertet wird. 
		
		\item \textbf{Wodurch unterscheidet sich Konsistenz von Konvergenz?} \\
			\(f_h\rightarrow f\) konvergiert für \( h \rightarrow 0\) falls \(f_h\) exakt ausgewertet wird. Computer müssen jedoch runden, wodurch wir nur noch von Konsistenz reden können wenn das Verfahren nicht stabil ist.
		
		\item \textbf{Was bedeutet der Begriff Stabilität?} \\
		S. 15\\
		Ein numerisches Verfahren \(f\) heißt stabil, falls bei der numerischen Auswertung \(\tilde{f}(x)\) des Verfahrens Fehler wie Rundungsfehler, Abbruchfehler und Verfahrensfehler von Teilschritten nicht	übermäßig verstärkt werden im Vergleich zu dem durch die relative Kondition \(\kappa_{rel}\) des Problems verursachten Fehler. \\
		Für die Differenz zwischen numerischer Auswertung \(\tilde{f}(x)\) mit obigen Fehlern und exakter Auswertung \(f(x)\) gilt dann
		\begin{align*}
			\norm{\tilde{f}(x)-f(x)}\leq \text{C}\cdot\kappa_{rel.}\cdot\abs{f(x)}\cdot\texttt{eps}, \quad \text{C} > 0 \text{ klein.}
		\end{align*}
		Ein numerisches Verfahren ist insbesondere dann stabil, wenn alle seine Teilschritte gut konditioniert sind.
		
	\end{enumerate}