% !TeX spellcheck = de_AT_frami
\section{Gewöhnliche Differentialgleichungen}
	\begin{enumerate}
		\item \textbf{Beschreiben Sie das \textit{explizite Euler-Verfahren} einer Dimension. Skizze!}
		\item \textbf{Wie schreibt man ein System von Differentialgleichungen zweiter Ordnung in ein System erster Ordnung um?}
		\item \textbf{Erklären Sie die Begriffe \textit{Runge-Kutta-Verfahren} und \textit{Butcher-Tableau}.}
		\item \textbf{Wie ist die Ordnung \(\mathbf{p}\) eines numerischen Verfahrens zur Lösung von Differentialgleichungen definiert?}
		\item \textbf{Was gilt für den \textit{lokalen} bzw. \textit{globalen Fehler} bei einem Verfahren mit Ordnung \(\mathbf{p}\)?}
		\item \textbf{Wie lässt sich die Ordnung \(\mathbf{p}\) experimentell feststellen?}
		\item \textbf{Was bedeutet der Begriff \textit{steife Differentialgleichung}?}
		\item \textbf{Es sei die Differentialgleichung \(\mathbf{y′=−200y, y(0)=1}\), gegeben. Wie groß darf man die Schrittweite \(\mathbf{h}\) maximal wählen, damit die numerische Lösung stabil bleibt, d.h. im Bereich zwischen −1 und 1 bleibt?}
		\item \textbf{Wie lässt sich der \textit{Fehler} \texttt{est} bei einer Schrittweitensteuerung schätzen?}
		\item \textbf{Wie bestimmt man bei einer Schrittweitensteuerung die \textit{optimale Schrittweite} \(\mathbf{h}_\texttt{opt}\)?}
	\end{enumerate}