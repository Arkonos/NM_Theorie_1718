% !TeX spellcheck = de_AT_frami
\section{Numerische Integration}
	\begin{enumerate}
		\item \textbf{Wie ist die LU-Zerlegung definiert?} \\
			Es sei A eine invertierbare \(n\times n\)-Matrix. Dann existieren drei ebenfalls invertierbare Matrizen und zwar eine Permutationsmatrix P, eine untere Dreieckmatrix L und eine obere Dreiecksmatrix U, d.h.
			\begin{align*}
				L = \begin{bmatrix}
					1 & 0 & \dots & 0 \\
					\ell_{21} & \ddots & \ddots & \vdots \\
					\vdots & \ddots & \ddots & 0 \\
					\ell_{n1} & \dots & \ell_{n,n-1} & 1
				\end{bmatrix}, \quad
				U = \begin{bmatrix}
					u_{11} & u_{12} & \dots & u_{1n}\\
					0 & \ddots & \ddots & \vdots \\
					\vdots & \ddots & \ddots & u_{n-1,n} \\
					0 & \dots & 0 & u_{nn}
				\end{bmatrix}
			\end{align*}
			mit \(u_{ii}\neq0,\,i=1,\dots,n\), sodass
			\begin{align*}
				PA=LU
			\end{align*}
			gilt. Diese Zerlegung heißt LU-Zerlegung. Die Multiplikation mit P von links entspricht einer Vertauschung der Zeilen von A wegen der Spaltenpivotsuche.
		
		\item \textbf{Was bedeutet Spaltenpivotsuche und wieso wird sie verwendet?} \\
			Für die LU-Zerlegung wählt man das betragsmäßig größte Element der Pivotspalte als Pivotelement. Dieser Vorgang wird als Spaltenpivotsuche bezeichnet. \\
			Im \(k\).ten Schritt der LU-Zerlegung heißt dies: Suche das betragsgrößte Element in der \(k\)-ten Spalte in den Zeilen \(k,k+1,\dots,n\)!
		
		\item \textbf{Wie erhält man die Permutationsmatrix bzw. den Permutationsvektor?} \\
		Permutationsmatrix P
			\begin{itemize}
				\item Starte mit der Einheitsmatrix
				\item Führe an dieser Matrix ebenfalls die entsprechend der Pivotsuche beim Gaußschen Algorithmus notwendige Zeilenvertauschung durch.
			\end{itemize}
		Permutationsvektor IP mit Länge \(n\)
			\begin{itemize}
				\item \(\text{IP}[n]=(-1)^{\text{Anzahl Vertauschungen}}\)
				\item \(\text{IP}[k]=m\) \\
					Im \(k\)-ten Eliminationsschritt wird die Zeile \(m\) mit der aktuellen Zeile \(k\) vertauscht.
				\item Alternativ: Man erstellt Vektor \(\begin{bmatrix}
				1 & 2 & \cdots & n
				\end{bmatrix}^\text{T}\) und wendet darauf die Vertauschen an.
			\end{itemize}
		
		\item \textbf{Wie löst man die LU-Zerlegung lineare Gleichungssysteme \(\mathbf{AX=b}\)?} \\
			\begin{enumerate}
				\item[Schritt 1:] Vertauschen der Elemente von b mit Hilfe von IP \(\Rightarrow\tilde{\text{b}}\)
				\item[Schritt 2:] Vorwärtssubstitution mit L und \(\tilde{\text{b}}\) \\
				\(\text{Ly}=\tilde{\text{b}}\Rightarrow\text{y}\)
				\item[Schritt 3:] Rückwärtssubstitution mit U und y \\
				\(\text{Ux=y}\Rightarrow \text{x}\)
			\end{enumerate}
		
		\item \textbf{Wie berechnet man \(\mathbf{\det A}\) mit Hilfe der LU-Zerlegung?} \\
			\begin{align*}
				\det{A}=u_{11}\cdot u_{22}\cdots u_{nn}\cdot\text{IP[n]}
			\end{align*}
		
		\item \textbf{Wie groß ist der Rechenaufwand der LU-Zerlegung und für das Auflösen eines linearen Gleichungssystems?}
		\begin{itemize}
			\item LU-Zerlegung: \(\frac{n^3}{3}+\mathcal{O}(n^2) \quad\) \text{Operationen}
			\item Auflösen l.Gls.: \(n^2+\mathcal{O}(n) \quad\) Operationen
		\end{itemize}
		
		\pagebreak
		\item \textbf{Wie ist die Cholesky-Zerlegung einer Matrix A definiert? Welche Eigenschaften muss A besitzen?} \\
			Cholesky-Zerlegung: A=CC\(^\text{T}\) wobei der Cholesky-Faktor C eine untere Dreiecksmatrix von A ist. \\
			Eigenschaften die A haben muss:
			\begin{itemize}
				\item reell: Die Koeffizienten von A sind reelle Zahlen;
				\item symmetrisch: \(\text{A}=\text{A}^\text{T}\);
				\item positiv definit: Für alle \(x\neq0\) gilt \(x^\text{T}Ax>0\).
			\end{itemize}
		
		\item \textbf{Wie berechnet man die Cholesky-Zerlegung?} \\
			Für alle Spalten \(k=1,\dots,n\) von links beginnend:
			\begin{itemize}
				\item[] Diagonalelement \(c_{kk}=\sqrt{a_{kk}-c_{k1}^2-\cdots-c_{k,k-1}^2}\)
				\item[] Für alle Elemente \(c_{ik},i>k,\) also unterhalb von \(c_{kk}\):\\
				\mbox{}\hspace{0.5cm}\(c_{ik}=(a_{ik}-c_{i1}c_{k1}-\cdots-c_{i,k-1}c_{k,k-1})/c_{kk}\)
			\end{itemize}
		
		\item \textbf{Wie groß ist der Rechenaufwand einer Cholesky-Zerlegung?} \\
			\(\frac{n^3}{6}+\mathcal{O}(n^2)\), Hälfte der LU-Zerlegung dank Symmetrie von A.
		
		\item \textbf{Wie löst man mit der Cholesky-Zerlegung lineare Gleichungssysteme \(\mathbf{Ax=b}\)?}
			\begin{align*}
				\text{Ax}=\text{b}&\Longleftrightarrow \text{C}\underbrace{\text{C}^\text{T}\text{x}}_\text{y}=\text{b} \\
				\text{Cy}&=\text{b} \Rightarrow \text{y} \\
				\text{C}^\text{T}\text{x}&=\text{y} \Rightarrow \text{x}
			\end{align*}
		
		\item \textbf{Wie ist die rationale Cholesky-Zerlegung einer Matrix A definiert? Welche Eigenschaften muss A besitzen?}
			\begin{align*}
				\text{A}=\text{LDL}^\text{T}
			\end{align*}
			Wobei L eie untere Dreiecksmatrix mit Einsen in der Diagonale und D eine Diagonalmatrix mit \(d_{ii}\neq0\) ist.\\
			A muss invertierbar und symmetrisch sein.
		
		\item \textbf{Wie berechnet man die rationale Cholesky-Zerlegung?} \\
			
		
		\item \textbf{Wie löst man mit der rationalen Cholesky-Zerlegung lineare Gleichungssysteme \(\mathbf{Ax=b}\)?}
			\begin{align*}
				\text{Ax}=\text{b}&\Longleftrightarrow \text{LDL}^\text{T}\text{x}=\text{b}
			\end{align*}
		
		\item \textbf{Wie ist die Matrixnorm allgemein definiert?} \\
			Gegeben sei ein Vektorraum \(V\). Eine Norm \(V\) ist eine Abbildung \(||\cdot||:V\rightarrow\mathbb{R}\) mit den folgenden Eigenschaften:
			\begin{itemize}
				\item[(i)] \(||x||\geq 0,\quad ||x||=0\Rightarrow x=0\),
				\item[(ii)] \(||\lambda x||=|\lambda|\cdot||x||\),
				\item[(iii)] \(||x+y||\leq||x||+||y||\). (Dreiecksungleichung)
			\end{itemize}
		
		\item \textbf{Wie berechnet man \(\mathbf{||\text{A}||_1}\), \(\mathbf{||\text{A}||_2}\) und \(\mathbf{||\text{A}||_\infty}\) ?} \\
			\begin{itemize}
				\item Es sei A eine \(m\times n\)-Matrix.
					Die Matrixnorm für die 1-Norm
					\begin{align*}
						||\text{A}||_1=\underset{1\leq j\leq n}{\max}\left( \sum_{i=1}^{m}|a_{ij}|\right) 
					\end{align*}
					ist die maximale Spaltenbetragssumme. Merkregel: 1 steht wie eine Spalte.
				\item Die Matrixnorm für die \(\infty\)-Norm
					\begin{align*}
						||\text{A}||_\infty=\underset{1\leq i\leq m}{\max}\left( \sum_{j=1}^{n}|a_{ij}|\right)
					\end{align*}
					ist die maximale Zeilenbetragssumme. Merkregel: \(\infty\) liegt wie eine Zeile.
				\item Die Matrixnorm für die 2-Norm (euklidische Norm):
					\begin{align*}
						||\text{A}||_2=\sqrt{\text{größere Eigenwerte von A}^\text{T}\text{A}}
					\end{align*}
			\end{itemize}		
			
		\item \textbf{Wie ist die Norm von \(\mathbf{A^{-1}}\) definiert?} \\
			\begin{align*}
				||\text{A}^{-1}||:=\left(\underset{x\neq0}{\text{inf}}\frac{||\text{Ax}||}{||\text{x}||}\right)^{-1} = \left(\underset{||\text{y}||=1}{\text{inf}}||\text{Ay}||\right)^{-1}
			\end{align*}
		
		\item \textbf{Wie ist die Kondition einer linear Abbildung bzw. einer Matrix A definiert?}
			\begin{align*}
				\kappa(\text{A}):=||\text{A}||\cdot||\text{A}^{-1}||
			\end{align*}
		
		\item \textbf{Was gilt für den relativen Fehler der Lösung eines linearen Gleichungssystems?}
			\begin{align*}
				\frac{||\tilde{\text{x}}-\text{x}||}{||\text{x}||}\leq \frac{\kappa(\text{A})}{1-\varepsilon_\text{A}\cdot\kappa(\text{A})}(\varepsilon_\text{A}+\varepsilon_\text{b})
			\end{align*}
		
		\item \textbf{Welche Eigenschaften besitzt die Kondition einer Matrix?}
			\begin{enumerate}
				\item[(1)] \(\kappa(\text{A})\geq1\).
				\item[(2)] Für \(\alpha\neq0\) ist \(\kappa(\alpha\text{A})=\kappa(\text{A})\)
				\item[(3)] \(\kappa(\text{A})=\frac{\sup_{||y||_2=1}||\text{Ay}||_2}{\inf_{||y||_2=1}||\text{Ay}||_2}=
				\frac{\sqrt{\lambda_{\max}}}{\sqrt{\lambda_{\min}}}\)
			\end{enumerate}
		
		\item \textbf{Welche Matrizen haben eine große/kleine Kondition?} \\
		Groß: Hilbertmatrix, Vandermonde-Matrix \\
		Klein: \(\text{A}=\alpha\text{I}, \quad \kappa(\text{A})=\kappa(\alpha\text{I})=\kappa(\text{I})=1\)\\
		\mbox{}\hspace{0.88cm} Für eine orthogonale Matrix Q gilt \(\kappa_2(\text{Q})=1\)
		
		\item \textbf{Was ist die Kondition der Einheitsmatrix?}
			\begin{align*}
				\kappa(\text{I})=1
			\end{align*}
		
		\item \textbf{Wie lässt sich die Kondition einer Matrix geometrisch interpretieren?} \\
			Es sei \(\varepsilon=\left\lbrace \text{A}\cdot\text{y}:||\text{y}||_2=1 \right\rbrace \) jene Ellipse, die durch Anwendung der Matrix auf A auf alle Vektoren mit Länge 1 erzeugt wird. Dann gilt:\\
			\begin{centering}
				\(\kappa(\text{A}) \) = Umkreisradius
			\end{centering}
		
		\item \textbf{Wie lässt sich die Kondition eines linearen Gleichungssystems geometrisch interpretieren?} \\
		
		\item \textbf{Wie ist die QR-Zerlegung definiert? Wozu wird sie verwendet?} \\
		
	\end{enumerate}