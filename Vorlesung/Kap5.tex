% !TeX spellcheck = de_AT_frami
\section{Numerische Integration}
	\begin{enumerate}
		\item \textbf{Wie ist die LU-Zerlegung definiert?} \\
		
		\item \textbf{Was bedeutet Spaltenpivotsuche und wieso wird sie verwendet?} \\
		
		\item \textbf{Wie erhält man die Permutationsmatrix bzw. den Permutationsvektor?} \\
		
		\item \textbf{Wie löst man die LU-Zerlegung lineare Gleichungssysteme \(\mathbf{AX=b}\)?} \\
		
		\item \textbf{Wie berechnet man \(\mathbf{\det A}\) mit Hilfe der LU-Zerlegung?} \\
		
		\item \textbf{Wie groß ist der Rechenaufwand der LU-Zerlegung und für das Auflösen eines linearen Gleichungssystems?} \\
		
		\item \textbf{Wie ist die Cholesky-Zerlegung einer Matrix A definiert? Welche Eigenschaften muss A besitzen?} \\
		
		\item \textbf{Wie berechnet man die Cholesky-Zerlegung?} \\
		
		\item \textbf{Wie groß ist der Rechenaufwand einer Cholesky-Zerlegung?} \\
		
		\item \textbf{Wie löst man mit der Cholesky-Zerlegung lineare Gleichungssysteme \(\mathbf{AX=b}\)?} \\
		
		\item \textbf{Wie ist die rationale Cholesky-Zerlegung einer Matrix A definiert? Welche Eigenschaften muss A besitzen?} \\
		
		\item \textbf{Wie berechnet man die rationale Cholesky-Zerlegung?} \\
		
		\item \textbf{Wie löst man mit der rationalen Cholesky-Zerlegung lineare Gleichungssysteme \(\mathbf{AX=b}\)?} \\
		
		\item \textbf{Wie ist die Matrixnorm allgemein definiert?} \\
		
		\item \textbf{Wie berechnet man \(\mathbf{||\text{A}||_1}\), \(\mathbf{||\text{A}||_2}\) und \(\mathbf{||\text{A}||_\infty}\) ?} \\
		
		\item \textbf{Wie ist die Norm von \(\mathbf{A^{-1}}\) definiert?} \\
		
		\item \textbf{Wie ist die Kondition einer linear Abbildung bzw. einer Matrix A definiert?} \\
		
		\item \textbf{Was gilt für den relativen Fehler der Lösung eines linearen Gleichungssystems?} \\
		
		\item \textbf{Welche Eigenschaften besitzt die Kondition einer Matrix?} \\
		
		\item \textbf{Welche Matrizen haben eine große/kleine Kondition?} \\
		
		\item \textbf{Was ist die Kondition der Einheitsmatrix?} \\
		
		\item \textbf{Wie lässt sich die Kondition einer Matrix geometrisch interpretieren?} \\
		
		\item \textbf{Wie lässt sich die Kondition eines linearen Gleichungssystems geometrisch interpretieren?} \\
		
		\item \textbf{Wie ist die QR-Zerlegung definiert? Wozu wird sie verwendet?} \\
		
	\end{enumerate}