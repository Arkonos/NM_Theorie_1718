\section{Interpolation}
	\begin{enumerate}
		\item \textbf{Wie werden die dividierten Differenzen berechnet?} \\
		
		\item \textbf{Wie ist das Newtonsche Interpolationspolynom definiert?} \\
		
		\item \textbf{Wie wird mit dem Hornerschema ein Polynom $\pmb{p(x)=a_0+a_1x+\dots +a_nx^n}$ ausgewertet?} \\
		
		\item \textbf{Wie wird mit dem Hornerschema ein Newtonsches Interpolationspolynom ausgewertet?} \\
		
		\item \textbf{Wie sind die Lagrange-Polynome definiert? Welche Eigenschaften haben sie?} \\
		
		\item \textbf{Wie wird mit Hilfe der Lagrange-Polynome das Langrangesche Interpolationspolynom berechnet?} \\
		
		\item \textbf{Erklären Sie die Begriffe Datenfehler, Verstärkungsfaktor, Lebesgue-Funktion und Lebesgue-Konstante in Zusammenhang mit der Polynominterpolation. Was ist die Kondition der Polynominterpolation?} \\
		
		\item \textbf{Was besagt der Satz über den Fehler des Interpolationspolynoms? Wie ist der Verfahrensfehler definiert?} \\
		
		\item \textbf{Wie sind die Tschebyscheff-Polynome definiert? Welche Eigenschaften haben sie?} \\
		
		\item \textbf{Wie berechnet man die Knoten für die Tschebyscheff-Interpolationspolynome im Intervall [−1, 1] bzw. [a, b]? Welche Vorteile hat die Verwendung von Tschebyscheff-Knoten im Vergleich zu äquidistanten Stützstellen.} \\
		
		\item \textbf{Wie lässt sich das dividierte Differenzenschema und das Newtonsche Interpolationspolynom verallgemeinern, falls in den Stützstellen auch noch Ableitungen vorgegeben sind?} \\
		
		\item \textbf{Wie wird mit stückweise konstanten Funktionen interpoliert?} \\
		
		\item \textbf{Wie wird mit stetigen, stückweise linearen Funktionen interpoliert?} \\
		
		\item \textbf{Was sind Hutfunktionen und welche Eigenschaften haben sie?} \\
		
		\item \textbf{Was für Eigenschaften besitzen kubische Splines? Was für Typen von kubischen Splines gibt es?} \\
		
		\item \textbf{Wieso ist es besser durch viele Punkte einen kubischen Spline zu legen, statt ein Interpolationspolynom zu verwenden?} \\
		
		\item \textbf{Wie wird auf einem rechteckigen Gitter zweidimensional interpoliert?} \\
		
		\item \textbf{Wie wird die zweidimensionale, stetige, stückweise lineare Interpolierende auf einem rechteckigen Gitter bestimmt?} \\
		
	\end{enumerate}