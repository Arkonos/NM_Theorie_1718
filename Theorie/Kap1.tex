\section{Zahldarstellung, Rundung und Fehler}
	\begin{enumerate}
		\item \textbf{Wie werden ganze Zahlen binär abgespeichert?}\\
			S. 2
			\begin{align*}
				b_{N-1}b_{N-2}\dots b_{1}b_{0} \cong b = \sum_{j=0}^{N-1}b_j2^j, \quad b_j\in\{0,1\}
			\end{align*}
			Beispiel: $23_{10}$
			\begin{align*}
					10101_2 &= 1\cdot2^4+0\cdot2^3+0\cdot2^2+1\cdot2^1+1\cdot2^0 \\
					      &= 16+0+0+2+1 = 19_{10}
			\end{align*} \vspace{-1cm}
			\begin{figure}[htbp]
				\centering
				\begin{minipage}{0.3\textwidth}
					\centering
					$$\begin{array}{rrrrrr}
					\multirow{2}{*}{:2} & 19 & 9 & 4 & 2 & 1 \\
					\cline{2-6}
					 \rule{0pt}{2.5ex} & 1 &  1 & 0 & 0 & 1
					\end{array}$$
				\end{minipage}\hspace{-0.5cm}
				\begin{minipage}{0.01\textwidth}
					\centering
					$\rightarrow$
				\end{minipage}\hspace{-0.7cm}
				\begin{minipage}{0.2\textwidth}
					\centering
					$10011_2$
				\end{minipage}
			\end{figure}
			
			
		\item \textbf{Wie werden Gleitpunktzahlen (doppelte Genauigkeit) binär abgespeichert?}\\
		S. 3
			\begin{align*}
				x &= (-1)^s \cdot m \cdot 2^e \\
				x &\cong s \;\;\; e_{11}e_{10}\dots e_0 \;\;\; (m_0).m_1m_2\dots m_{51}
			\end{align*}\vspace{-0.5cm}
			\begin{table}[htbp]
				\centering
				\begin{tabular}[htpb]{rll}
					s \; \dots \!\!\! & Vorzeichenbit &$\in \{0,1\}$\\
					m \; \dots \!\!\! & Mantisse & Normiert, $m_0\overset{!}{=}1$ wird weggelassen\\
					e \; \dots \!\!\! & Exponent & nach Abzug von b$\dots$Bias 
				\end{tabular}
			\end{table}
		
		\item \textbf{Wie werden Gleitpunktzahlen gerundet?} \\
		S. 6 \\
		Round to the nearest even.
		\item \textbf{Wie ist der relative Rundungsfehler definiert?} \\
		S. 6
		\begin{align*}
			\frac{|rd(x)-x|}{|x|}\leq \frac{2^{-M-1}\cdot2^e}{a\cdot 2^e}\underset{a\in[1,2)}{\leq}2^{-M-1}=:\texttt{eps}
		\end{align*}
		
		\item \textbf{Wie groß ist die relative Maschinengenauigkeit \texttt{eps} für doppelt genaue Gleitpunktzahlen?\\
					Wie kann man \texttt{eps} experimentell bestimmen?} \\
				
				
		\item \textbf{Was ist die relative/absolute Kondition eines Problems?} \\
		
		
		\item \textbf{Was bedeuten die Begriffe Konsistenz und Konsistenzordnung?} \\
		
		
		\item \textbf{Wodurch unterscheidet sich Konsistenz von Konvergenz?} \\
		
		
		\item \textbf{Was bedeutet der Begriff Stabilität?} \\
		S. 15\\
		Ein numerisches Verfahren $f$ heißt stabil, falls bei der numerischen Auswertung $f''(x)$ des Verfahrens Fehler wie Rundungsfehler, Abbruchfehler und Verfahrensfehler von Teilschritten nicht	übermäßig verstärkt werden im Vergleich zu dem durch die relative Kondition $\kappa_{rel}$ des Problems
		verursachten Fehler.
		
	\end{enumerate}